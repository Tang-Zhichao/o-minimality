\documentclass{amsart}

\usepackage[style=alphabetic,backref, giveninits, maxbibnames=10]{biblatex}
\addbibresource{o-minimal.bib}

\usepackage{graphicx} 
\usepackage{xcolor}
\usepackage{amsmath,amsfonts,amssymb,mathtools,amsthm}
\usepackage{enumitem}
\usepackage{hyperref}
\usepackage{xurl}

\usepackage{tikz}
\usetikzlibrary{cd}

\newtheorem{theorem}{Theorem}[section]
\newtheorem{lemma}[theorem]{Lemma}
\newtheorem{proposition}[theorem]{Proposition}
\newtheorem{corollary}[theorem]{Corollary}
\newtheorem{construction}[theorem]{Construction}

\theoremstyle{definition}
\newtheorem{definition}[theorem]{Definition}
\newtheorem{remark}[theorem]{Remark}
\newtheorem{example}[theorem]{Example}
\numberwithin{equation}{section}

\newcommand{\definable}{\mathrm{def}}
\newcommand{\analytic}{\mathrm{an}}

\title{Notes on o-minimality}
\author{Tang Zhichao}
\date{\today}

\begin{document}

\maketitle

\section{Basic model theory}
Materials follow from \cite{zbMATH01821671,zbMATH01160037},
mainly \cite{zbMATH01821671}.
\subsection{Languages and Structures}
In mathematical logic, 
we use first-order languages to describe mathematical structures. 
Intuitively, a structure is a set that we wish to study equipped with a collection of distinguished functions, relations, and elements. 
We then choose a language where we can talk about the distinguished functions, relations, and elements and nothing more.

\begin{definition}
A language $\mathcal{L}$ is given by specifying the following data:
\begin{enumerate}[label = {(\roman*)}]
    \item a set of function symbols $\mathcal{F}$ and positive integers $n_f$ for each $f \in \mathcal{F}$;
    \item a set of relation symbols $\mathcal{R}$ and positive integers $n_R$ for each $R \in \mathcal{R}$;
    \item a set of constant symbols $\mathcal{C}$.
\end{enumerate}
\end{definition}

\begin{definition}
An $\mathcal{L}$-structure $\mathcal{M}$ is given by the following data:
    \begin{enumerate}[label = {(\roman*)}]
    \item a nonempty set $M$ called the universe, domain, or underlying set of $\mathcal{M}$;
    \item a function $f^{\mathcal{M}} : M^{n_f} \to M$ for each $f \in \mathcal{F}$;
    \item a set $R^{\mathcal{M}} \subseteq M^{n_R}$ for each $R \in \mathcal{R}$;
    \item an element $c^{\mathcal{M}} \in M$ for each $c \in \mathcal{C}$.
    \end{enumerate}
\end{definition}

\begin{definition}
    Suppose that $\mathcal{M}$ and $\mathcal{N}$ are $\mathcal{L}$-structures with universes $M$ and $N$, respectively.
    An \emph{$\mathcal{L}$-embedding} is a one-to-one (injective) map $\eta: M \to N$ that preserves the interpretation of all of the symbols of $\mathcal{L}$.
    \begin{enumerate}[label = {(\roman*)}]
        \item $\eta(f^{\mathcal{M}}(a_1,\dots,a_{n_f})) = f^{\mathcal{N}}(\eta(a_1),\dots,\eta(a_{n_f}))$ for all $f\in \mathcal{F}$ and $a_1,\dots,a_{n_f} \in M$;
        \item $(a_1,\dots,a_{n_R}) \in R^{\mathcal{M}}$ if and only if $(\eta(a_1),\dots,\eta(a_{n_R})) \in R^{\mathcal{N}}$ for all $R \in \mathcal{R}$ and $a_1,\dots,a_{n_R} \in M$;
        \item $\eta(c^{\mathcal{M}}) = c^{\mathcal{N}}$ for all $c \in \mathcal{C}$.
    \end{enumerate}
    A bijective $\mathcal{L}$-embedding is called an \emph{$\mathcal{L}$-isomorphism}.
    If $M \subseteq N$ and the inclusion map is an $\mathcal{L}$-embedding,
    $\mathcal{M}$ is a \emph{substructure} of $\mathcal{N}$ or that
    $\mathcal{N}$ is an \emph{extension} of $\mathcal{M}$.
\end{definition}

The \emph{cardinality} of $\mathcal{M}$ is $|M|$,
the cardinality of the universe of $\mathcal{M}$.
If $\eta: \mathcal{M} \to \mathcal{N}$ is an embedding then 
the cardinality of $\mathcal{N}$ is at least the cardinality of $\mathcal{M}$.

\begin{definition}
    The set of \emph{$\mathcal{L}$-terms} is the smallest set $\mathcal{T}$ such that
    \begin{enumerate}[label = {(\roman*)}]
        \item $c\in \mathcal{T}$ for each constant symbol $c \in \mathcal{C}$,
        \item each variable symbol $v_i \in \mathcal{T}$ for $i = 1,2,\dots$, and
        \item if $t_1,\dots,t_{n_f} \in \mathcal{T}$ and $f \in \mathcal{F}$, then $f(t_1,\dots,t_{n_f}) \in \mathcal{T}$.
    \end{enumerate}
\end{definition}

\begin{definition}
    $\phi$ is an \emph{atomic $\mathcal{L}$-formula} if $\phi$ is either
    \begin{enumerate}[label= {\roman*)}]
        \item $t_1 = t_2$, where $t_1$ and $t_2$ are terms, or
        \item $R(t_1,\dots,t_{n_R})$, where $R \in \mathcal{R}$ and $t_1,\dots,t_{n_R}$ are terms.
    \end{enumerate}
    The set of \emph{$\mathcal{L}$-formulas} is the smallest set $\mathcal{W}$ containing the atomic formulas such that 
    \begin{enumerate}[label= {\roman*)}]
        \item if $\phi$ is in $\mathcal{W}$, then $\neg\phi$ is in $\mathcal{W}$,
        \item if $\phi$ and $\psi$ are in $\mathcal{W}$, then $(\phi \land \psi)$ and $(\phi \lor \psi)$ are in $\mathcal{W}$, and 
        \item if $\psi$ is in $\mathcal{W}$, then $\exists v_i\, \phi$ and $\forall v_i\, \phi$ are in $\mathcal{W}$.
    \end{enumerate}
\end{definition}

\subsection{Basic Techniques and Important Results}

\begin{theorem}[G\"odel's Completeness Theorem]
    Let $T$ be an $\mathcal{L}$-theory and $\phi$ an $\mathcal{L}$-sentence, then $T \models \phi$ if and only if $T\vdash \phi$.
\end{theorem}

\begin{theorem}[Compactness Theorem]
    $T$ is satisfiable if and only if every finite subset of $T$ is satisfiable.
\end{theorem}

\begin{theorem}[Vaught's Test]
    Let $T$ be a satisfiable theory with no finite models that is 
    $\kappa$-categorical for some infinite cardinal $\kappa \ge |\mathcal{L}|$.
    Then $T$ is complete.
\end{theorem}

Some abbreviation for common theories:
\begin{enumerate}[label = {}]
    \item \textbf{DLO}: dense linear order;
    \item \textbf{ACF}: algebraic closed field;
    \item \textbf{RCF}: real closed field.
\end{enumerate}

\begin{theorem}[Hilbert's 17th Problem]
    If $f$ is a positive semi-definite rational function over a real closed field $F$,
    then $f$ is a sum of squares of rational functions.
\end{theorem}

\begin{corollary}
    $\mathbf{ACF}$ is a strongly minimal theory.
\end{corollary}

\begin{corollary}
    $\mathbf{RCF}$ is an o-minimal theory.
\end{corollary}

We work with a fixed but arbitrary o-minimal structure $(R,<,\mathcal{S})$.

\begin{theorem}[Monotonicity Theorem]
    Let $f: (a,b)\to R$ be a definable function on the interval $(a,b)$.
    Then there are points $a_1 < \cdot < a_k$ in $(a,b)$ such that on each subinterval $(a_j,a_{j+1})$,
    with $a_0 = a$, $a_{k+1} = b$,
    the function is either constant, or strictly monotone and continuous.
\end{theorem}

\begin{corollary}
    Let $f: (a,b) \to R$ be definable.
    Then for each $c\in (a,b)$ the limit $\lim_{x\uparrow c}f(x)$
    and $\lim_{x\downarrow c}f(x)$ exist in $R_{\infty}$.
    Also the limits $\lim_{x\uparrow b}f(x)$ and $\lim_{x\downarrow a}f(x)$ exist in $R_{\infty}$.
\end{corollary}

\begin{corollary}
    Let $f: [a,b]\to R$ be continuous and definable.
    Then $f$ takes a maximum and a minimum value on $[a,b]$.
\end{corollary}

\begin{lemma}[Finiteness Lemma]
    Let $A \subset R^2$ be definable and suppose that for each $x\in R$
    the fiber $A_x \coloneq \{y\in R\colon (x,y)\in A\}$ is finite.
    Then there is $N \in \mathbb{N}$ such that $A_x \le N$ for all $x\in R$.
\end{lemma}

A decomposition $\mathcal{D}$ of $R^m$ is said to \textbf{partition}
a set $S \subseteq R^m$ if each cell in $\mathcal{D}$ is
either part of $S$ or disjoint from $S$,
in other words, if $S$ is a union of cells in $\mathcal{D}$.
\begin{theorem}[Cell Decomposition]
    \hfill
    \begin{enumerate}[label = {$\mathrm{(\Roman*_{m})}$}]
        \item Given any definable sets $A_1,dots,A_k \subset R^m$ there is a decomposition of $R^m$ partitioning each of $A_1,\dots,A_k$.
        \item For each definable function $f: A \to R$, $A \subseteq R^m$,
        there is a decomposition $\mathcal{D}$ of $R^m$ partitioning $A$ such that 
        the restriction $f|B: B \to R$ to each cell $B \in \mathcal{D}$ with $B \subseteq A$ is continuous.
    \end{enumerate}
\end{theorem}

\begin{theorem}
    The theory $\mathbb{R}_{\exp}$ is model-complete and o-minimal.
\end{theorem}

\section{O-minimal GAGA Results}
O-minimal GAGA theorems are from \cite{zbMATH07662555},
which are used in the proof of the boundedness in \cite{arXiv:2507.00973,arXiv:2508.19215}
to adapt to the non-algebraic period domains from moduli spaces of abelian varieties.

\begin{theorem}[{Definable GAGA, \cite[Theorem 2.1]{arXiv:2508.19215},\cite[Theorem 1.4]{zbMATH07662555}}]
    Le $X$ be an algebraic space and $X^{\definable}$ the associated definable analytic space.
    The definabilization functor $\mathrm{Coh}(X) \to \mathrm{Coh}(X^{\definable})$ is fully faithful, exact and 
    its essential image if closed under subobjects and quotients.
\end{theorem}

\begin{theorem}[{Definable images, \cite[Theorem 2.2]{arXiv:2508.19215},\cite[Theorem 1.3]{zbMATH07662555}}]
    Let $X$ be an algebraic space and 
    $\phi: X^{\definable} \to \mathcal{Z}$ a proper morphism of definable analytic spaces.
    Then there is a factorization
    \[\begin{tikzcd}
    X^{\definable}\ar[rr,"\phi"]\ar[dr,"f^{\definable}"'] & & \mathcal{Z} \\
    & Y^{\definable}\ar[ur,"\iota"'] &
    \end{tikzcd}\]
    where $f: X\to Y$ is a proper dominant morphism of algebraic spaces and 
    $\iota: Y^{\definable} \to \mathcal{Z}$ is a closed embedding of definable analytic spaces.
    Moreover, $f$ is uniquely determined as a morphism with fixed source.
\end{theorem}

\begin{proposition}[{\cite[Proposition 2.45]{zbMATH07662555}}]
    Let $X$ be a definable complex analytic space and 
    $\mathcal{Y} \subset X$ a closed definable complex analytic subset.
    Then $\mathcal{Y}$ canonically has the structure of a reduced closed definable complex analytic subspace $Y \subset X$.
\end{proposition}

\begin{proposition}[{\cite[Proposition 2.55]{zbMATH07662555}}]
    Let $X$, $Y$, $Z$ be definable complex analytic spaces and
    suppose we have (solid) diagrams
    \[
    \begin{tikzcd}
        X \ar[r,"h"] \ar[d,"g"] & Y\ar[dl,"f",dotted] \\ Z 
    \end{tikzcd}
    \hspace*{30pt}
    \begin{tikzcd}
        X^{\analytic} \ar[r,"h^{\analytic}"] \ar[d,"g^{\analytic}"] & Y^{\analytic}\ar[dl,"\varphi"] \\ Z^{\analytic} 
    \end{tikzcd}
    \]
    such that $h$ is proper, surjective on points,
    and $\mathcal{O}_Y \to h_*\mathcal{O}_X$ is injective.
    Then a unique $f$ exists such that $f^{\analytic} = \varphi$.
\end{proposition}

\begin{theorem}[{\cite[Theorem 2.38]{zbMATH07662555}}]
    Let $X$ be a definable complex analytic space.
    Then $\mathcal{O}_X$ is a coherent $\mathcal{O}_X$-module.
\end{theorem}

\begin{theorem}[{\cite[Theorem 2.39]{zbMATH07662555}}]
    Let $X$ be a definable complex analytic space.
    Then the analytification functor 
    $(-)^{\analytic}: \mathbf{Coh}(X) \to \mathbf{Coh}(X^{\analytic})$ is exact and faithful.
\end{theorem}

\begin{corollary}[{\cite[Corollary 2.40]{zbMATH07662555}}]
    For $X$ a definable complex analytic space,
    a sequence $M' \to M \to M''$ of coherent $\mathcal{O}_X$-modules
    is exact if and only if it is exact on stalks (or even analytic stalks).
\end{corollary}

\begin{theorem}[{\cite[Theorem 3.1]{zbMATH07662555}}]
    Let $X$ be an algebraic space and 
    $(-)^{\definable} : \mathbf{Coh}(X) \to \mathbf{Coh}(X^{\definable})$ the definabilization functor.
    Then 
    \begin{enumerate}[label = {(\arabic*)}]
        \item $(-)^{\definable}$ is fully faithful and exact.
        \item The essential image of $(-)^{\definable}$ is closed under taking subobjects and quotients.
    \end{enumerate}
\end{theorem}

\begin{theorem}[{Definable Chow theorem, \cite[Theorem 2.23]{zbMATH07662555}, Peterzil--Strarchenko \cite[Theorem 2.14]{zbMATH05364146}}]
    Let $Y$ be a reduced algebraic space and 
    $\mathcal{X} \subset Y^{\definable}$ a closed definable complex analytic subset.
    Then $\mathcal{X}$ is algebraic.
\end{theorem}

\begin{corollary}[{\cite[Corollary 3.10]{zbMATH07662555}}]
    Let $Y$ be an algebraic space and $\mathcal{X} \subset Y^{\definable}$ a closed definable complex analytic subspace.
    Then $\mathcal{X}$ is (uniquely) the definabilization of an algebraic subspace.
\end{corollary}

\begin{corollary}[{\cite[Corollary 3.11]{zbMATH07662555}}]
    Let $X$, $Y$ be algebraic spaces.
    Then any morphism $X^{\definable} \to Y^{\definable}$ of definable complex analytic space is (uniquely) the definabilization of an algebraic morphism.
\end{corollary}

\begin{proposition}[{\cite[Proposition 4.5]{zbMATH07662555}}]
    Let $f: W \to Z$ be a proper dominant morphism of algebraic spaces.
    Suppose we have an algebraic square-zero thickening
    $W \to W'$, a definable closed immersion $Z^{\definable} \to \mathcal{Z}'$,
    and a morphism $\varphi': W^{\definable}\to \mathcal{Z}'$ which fits into a commutative diagram
    \[\begin{tikzcd}
        W^{\definable} \ar[r]\ar[d,"f^{\definable}"'] & W^{'\definable}\ar[d,"\varphi'"] \\
        Z^{\definable} \ar[r] & \mathcal{Z}'
    \end{tikzcd}\]
    Then the following are uniquely defined:
    an algebraic square-zero thickening
    $Z \to Z''$,
    a definable closed immersion $Z^{''\definable} \to \mathcal{Z}'$,
    and a (proper) dominant morphism $f': W' \to Z''$ of algebraic spaces,
    such that we have commutative diagrams
    \[
    \begin{tikzcd}
        W\ar[r]\ar[d,"f"] & W'\ar[d,"f'"] \\
        Z \ar[r] & Z''
    \end{tikzcd}
    \hspace{40pt}
    \begin{tikzcd}
        W^{'\definable}\ar[dd,"f^{'\definable}"]\ar[rd,"\varphi'"] & \\
        & \mathcal{Z}' \\
        Z^{''\definable}\ar[ur] &
    \end{tikzcd}
    \]
\end{proposition}

\begin{proposition}[{\cite[Proposition 4.6]{zbMATH07662555}}]
    Let $X$ be a smooth algebraic space,
    $\mathcal{U}$ a smooth definable complex analytic space,
    and $\varphi: X^{\definable} \to \mathcal{U}$ a smooth proper definable analytic morphism.
    Then $\varphi: X^{\definable} \to \mathcal{U}$ is the definabilization of an algebraic morphism $f: X \to U$ and 
    the associated morphism $U \to \mathrm{Hilb}(X)$ is a closed embedding.
\end{proposition}

\begin{lemma}[{\cite[Lemma 4.11]{zbMATH07662555}}]
    Suppose $Z$ is an algebraic space,
    and $J$ is a coherent sheaf on $Z$.
    Let $R$ be a sheaf of rings on the {\'e}tale site $Z^{\text{\' et}}$ of $Z$ such that 
    \[
    0 \to J \to R \to \mathcal{O}_Z \to 0
    \]
    is a first order thickening.
    Then $(Z^{\text{\'et}},R)$ is an algebraic space.
\end{lemma}

\emergencystretch = 1em
\printbibliography

\end{document}