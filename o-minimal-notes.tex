\documentclass{amsart}

\usepackage[style=alphabetic,backref, giveninits, maxbibnames=10]{biblatex}
\addbibresource{o-minimal.bib}

\usepackage{graphicx}
\usepackage{xcolor}
\usepackage{amsmath,amsfonts,amssymb,mathtools,amsthm}
\usepackage{enumitem}
\usepackage{hyperref}
\usepackage{xurl}

\usepackage{tikz}
\usetikzlibrary{cd}

\usepackage{tcolorbox}

\newtheorem{theorem}{Theorem}[subsection]
\newtheorem{lemma}[theorem]{Lemma}
\newtheorem{proposition}[theorem]{Proposition}
\newtheorem{corollary}[theorem]{Corollary}
\newtheorem{construction}[theorem]{Construction}

\theoremstyle{definition}
\newtheorem{definition}[theorem]{Definition}
\newtheorem{remark}[theorem]{Remark}
\newtheorem{example}[theorem]{Example}
\numberwithin{equation}{section}

\newcommand{\definable}{\mathrm{def}}
\newcommand{\analytic}{\mathrm{an}}

\title{Notes on o-minimality}
\author{Tang Zhichao}
\date{\today}

\begin{document}

\maketitle

\section{Basic model theory}
Materials follow from \cite{zbMATH01821671,zbMATH01160037},
mainly \cite{zbMATH01821671}.
\subsection{Languages and Structures}
\subsubsection{Basic definitions}
In mathematical logic,
we use first-order languages to describe mathematical structures.
Intuitively, a structure is a set that we wish to study equipped with a collection of distinguished functions, relations, and elements.
We then choose a language where we can talk about the distinguished functions, relations, and elements and nothing more.

\begin{definition}
A language $\mathcal{L}$ is given by specifying the following data:
\begin{enumerate}[label = {(\roman*)}]
    \item a set of function symbols $\mathcal{F}$ and positive integers $n_f$ for each $f \in \mathcal{F}$;
    \item a set of relation symbols $\mathcal{R}$ and positive integers $n_R$ for each $R \in \mathcal{R}$;
    \item a set of constant symbols $\mathcal{C}$.
\end{enumerate}
\end{definition}

\begin{definition}
An $\mathcal{L}$-structure $\mathcal{M}$ is given by the following data:
    \begin{enumerate}[label = {(\roman*)}]
    \item a nonempty set $M$ called the universe, domain, or underlying set of $\mathcal{M}$;
    \item a function $f^{\mathcal{M}} : M^{n_f} \to M$ for each $f \in \mathcal{F}$;
    \item a set $R^{\mathcal{M}} \subseteq M^{n_R}$ for each $R \in \mathcal{R}$;
    \item an element $c^{\mathcal{M}} \in M$ for each $c \in \mathcal{C}$.
    \end{enumerate}
\end{definition}

\begin{definition}
    Suppose that $\mathcal{M}$ and $\mathcal{N}$ are $\mathcal{L}$-structures with universes $M$ and $N$, respectively.
    An \emph{$\mathcal{L}$-embedding} is a one-to-one (injective) map $\eta: M \to N$
    that preserves the interpretation of all of the symbols of $\mathcal{L}$.
    \begin{enumerate}[label = {(\roman*)}]
        \item $\eta(f^{\mathcal{M}}(a_1,\dots,a_{n_f})) = f^{\mathcal{N}}(\eta(a_1),\dots,\eta(a_{n_f}))$
          for all $f\in \mathcal{F}$ and $a_1,\dots,a_{n_f} \in M$;
        \item $(a_1,\dots,a_{n_R}) \in R^{\mathcal{M}}$ if and only if
          $(\eta(a_1),\dots,\eta(a_{n_R})) \in R^{\mathcal{N}}$ for all $R \in \mathcal{R}$ and $a_1,\dots,a_{n_R} \in M$;
        \item $\eta(c^{\mathcal{M}}) = c^{\mathcal{N}}$ for all $c \in \mathcal{C}$.
    \end{enumerate}
    A bijective $\mathcal{L}$-embedding is called an \emph{$\mathcal{L}$-isomorphism}.
    If $M \subseteq N$ and the inclusion map is an $\mathcal{L}$-embedding,
    $\mathcal{M}$ is a \emph{substructure} of $\mathcal{N}$ or that
    $\mathcal{N}$ is an \emph{extension} of $\mathcal{M}$.
\end{definition}

The \emph{cardinality} of $\mathcal{M}$ is $|M|$,
the cardinality of the universe of $\mathcal{M}$.
If $\eta: \mathcal{M} \to \mathcal{N}$ is an embedding then 
the cardinality of $\mathcal{N}$ is at least the cardinality of $\mathcal{M}$.

\begin{definition}
    The set of \emph{$\mathcal{L}$-terms} is the smallest set $\mathcal{T}$ such that
    \begin{enumerate}[label = {(\roman*)}]
        \item $c\in \mathcal{T}$ for each constant symbol $c \in \mathcal{C}$,
        \item each variable symbol $v_i \in \mathcal{T}$ for $i = 1,2,\dots$, and
        \item if $t_1,\dots,t_{n_f} \in \mathcal{T}$ and $f \in \mathcal{F}$, then $f(t_1,\dots,t_{n_f}) \in \mathcal{T}$.
    \end{enumerate}
\end{definition}

\begin{definition}
    $\phi$ is an \emph{atomic $\mathcal{L}$-formula} if $\phi$ is either
    \begin{enumerate}[label= {\roman*)}]
        \item $t_1 = t_2$, where $t_1$ and $t_2$ are terms, or
        \item $R(t_1,\dots,t_{n_R})$, where $R \in \mathcal{R}$ and $t_1,\dots,t_{n_R}$ are terms.
    \end{enumerate}
    The set of \emph{$\mathcal{L}$-formulas} is the smallest set $\mathcal{W}$ containing the atomic formulas such that 
    \begin{enumerate}[label= {\roman*)}]
        \item if $\phi$ is in $\mathcal{W}$, then $\neg\phi$ is in $\mathcal{W}$,
        \item if $\phi$ and $\psi$ are in $\mathcal{W}$, then $(\phi \land \psi)$ and $(\phi \lor \psi)$ are in $\mathcal{W}$, and 
        \item if $\phi$ is in $\mathcal{W}$, then $\exists v_i\, \phi$ and $\forall v_i\, \phi$ are in $\mathcal{W}$.
    \end{enumerate}
\end{definition}

A variable $v$ \emph{occurs freely} in a formula $\phi$
if it is not inside a $\exists v$ or $\forall v$ quantifier;
otherwise, we say that it is \emph{bound}.
A formula is a \emph{sentence} if it has no free variables.

\begin{definition}
    Let $\phi$ be a formula with free variables from 
    $\overline{v} = (v_{i_1},\dots,v_{i_m})$, and let
    $\overline{a} = (a_{i_1},\dots,a_{i_m})\in M^m$.
    Inductively define $\mathcal{M} \models \phi(\overline{a})$ as follows.
    \begin{enumerate}[label = {\roman*)}]
        \item If $\phi$ is $t_1. = t_2$, then $\mathcal{M} \models \phi(\overline{a})$
          if $t_1^{\mathcal{M}}(\overline{a}) = t_2^{\mathcal{M}}(\overline{a})$.
        \item If $\phi$ is $R(t_1,\dots,t_{n_R})$, then $\mathcal{M} \models \phi(\overline{a})$
          if $(t^{\mathcal{M}}_1(\overline{a}),\dots,t^{\mathcal{M}}_{n_R}(\overline{a})) \in R^{\mathcal{M}}$.
        \item If $\phi$ is $\neg \psi$, then $\mathcal{M} \models \phi(\overline{a})$ if $\mathcal{M} \not\models \psi(\overline{a})$.
        \item If $\phi$ is $(\psi \land \theta)$, then $\mathcal{M} \models \phi(\overline{a})$
          if $\mathcal{M} \models \psi(\overline{a})$ and $\mathcal{M} \models \theta(\overline{a})$.
        \item If $\phi$ is $(\psi \lor \theta)$, then $\mathcal{M} \models \phi(\overline{a})$
          if $\mathcal{M} \models \psi(\overline{a})$ or $\mathcal{M} \models \theta(\overline{a})$.
        \item If $\phi$ is $\exists v_j \psi(\overline{v},v_j)$, then $\mathcal{M} \models \phi(\overline{a})$
          if there is $b \in M$ such that $\mathcal{M} \models \psi(\overline{a},b)$.
        \item If $\phi$ is $\forall v_j \psi(\overline{v},v_j)$,
          then $\mathcal{M} \models \phi(\overline{a})$ if $\mathcal{M} \models \psi(\overline{a},b)$ for all $b \in M$.
    \end{enumerate}
    If $\mathcal{M} \models \phi(\overline{a})$ we say that
    $\mathcal{M}$ \emph{satisfies} $\phi(\overline{a})$ or $\phi(\overline{a})$ is \emph{true} in $\mathcal{M}$.
\end{definition}

\begin{proposition}
    Suppose that $\mathcal{M}$ is a substructure of $\mathcal{N}$,
    $\overline{a} \in M$, and $\phi(\overline{v})$ is a quantifier-free formula.
    Then, $\mathcal{M} \models \phi(\overline{a})$ if and only if $\mathcal{N} \models \phi(\overline{a})$.
\end{proposition}

\begin{definition}
    Two $\mathcal{L}$-structures $\mathcal{M}$ and $\mathcal{N}$ are elementarily equivalent
    and denote $\mathcal{M} \equiv \mathcal{N}$ if
    \[
    \mathcal{M} \models \phi \text{ if and only if }\mathcal{N} \models \phi
    \]
    for all $\mathcal{L}$-sentences $\phi$.
\end{definition}

\subsubsection{Theories and logical consequence}
Let $\mathrm{Th}(\mathcal{M})$, the \emph{full theory} of $\mathcal{M}$,
be the set of $\mathcal{M}$-sentences $\phi$ such that $\mathcal{M} \models \phi$.

$\mathcal{M} \equiv \mathcal{N}$ if and only if $\mathrm{Th}(\mathcal{M}) = \mathrm{Th}(\mathcal{N})$.

\begin{theorem}
    Suppose that $j; \mathcal{M} \to \mathcal{N}$ is an isomorphism.
    Then, $\mathcal{M} \equiv \mathcal{N}$.
\end{theorem}

Let $\mathcal{L}$ be a language.
An $\mathcal{L}$-theory $T$ is a set of $\mathcal{L}$-sentences.

$\mathcal{M}$ is a \emph{model} of $T$ and denote $\mathcal{M} \models T$ if $\mathcal{M} \models \phi$ for all sentences $\phi \in T$.

A theory is \emph{satisfiable} if it has a model.

A class of $\mathcal{L}$-structures $\mathcal{K}$ is an \emph{elementary class}
if there is an $\mathcal{L}$-theory $T$ such that $\mathcal{K} = \{ \mathcal{M}:\mathcal{M} \models T \}$.
The sentences in $T$ are called \emph{axioms} for the elementary class.

\begin{definition}
    Let $T$ be an $\mathcal{L}$-theory and $\phi$ an $\mathcal{L}$-sentence.
    $\phi$ is a \emph{logical consequence} of $T$ and denote $T \models \phi$ if $\mathcal{M} \models \phi$
    whenever $\mathcal{M} \models T$.
\end{definition}

\subsubsection{Definability and Interpretability}
\begin{definition}
    Let $\mathcal{M} = (M,\dots)$ be an $\mathcal{L}$-structure.
    $X \subseteq M^n$ is \emph{definable} if and only if there is an $\mathcal{L}$-formula $\phi(v_1,\dots,v_n,w_1,\dots,w_m)$
    and $\overline{b} \in M^m$ such that $X = \{ \overline{a} \in M^n : \mathcal{M} \models \phi(\overline{a},\overline{b}) \}$.
    $\phi(\overline{v},\overline{b})$ \emph{defines} $X$.
    $X$ is \emph{$A$-definable }or \emph{definable over $A$} if there is a formula $\psi(\overline{v},w_1,\dots,w_l)$ and
    $\overline{b} \in A^l$ such that $\psi(\overline{v},\overline{b})$ defines $X$.
\end{definition}

\begin{lemma}
    Let $\mathcal{L}_r$ be the language of ordered rings and
    $(\mathbb{R}, +, -, \dot, <, 0, 1)$ be the ordered field of real numbers.
    Suppose that $X \subset \mathbb{R}^n$ is $A$-definable.
    Then, the topological closure of $X$ is also $A$-definable.
\end{lemma}

\begin{proposition}[Concrete Characterization of Definable Sets]
    Let $\mathcal{M}$ be an $\mathcal{L}$-structure.
    Suppose that $D_n$ is a collection of subsets of $M^n$ for all $n \ge 1$ and
    $\mathcal{D} = (D_n : n \ge 1)$ is the smallest collection such that:
    \begin{enumerate}[label = {\roman*)}]
        \item $M^n \in D_n$;
        \item for all $n$-ary function symbols $f$ of $\mathcal{L}$, the graph of $f^{\mathcal{M}}$ is in $D_{n+1}$;
        \item for all $n$-ary relation symbols $R$ of $\mathcal{L}$, $R^{\mathcal{M}} \in D_n$;
        \item for all $i$, $j\le n$, $\{(x_1,\dots,x_n)\in M^n : x_i = x_j\} \in D_n$;
        \item if $X \in D_n$, then $M \times X \in D_{n+1}$;
        \item each $D_n$ is closed under complement, union, and intersection;
        \item if $X \in D_{n+1}$ and $\pi: M^{n+1} \to M^n$ is the projection map $(x_1,\dots,x_{n+1}) \mapsto (x_1,\dots,x_n)$,
          then $\pi(X) \in D_n$;
        \item if $X \in D_{n+m}$ and $b \in M^m$, then $\{a\in M^n: (a,b) \in X \} \in D_n$.
    \end{enumerate}
    Then, $X \subseteq M^n$ is definable if and only if $X \in D_n$.
\end{proposition}

\begin{proposition}
    Let $\mathcal{M}$ be an $\mathcal{L}$-structure.
    If $X \subset M^n$ is $A$-definable,
    then every $\mathcal{L}$-automorphism of $\mathcal{M}$ that fixes $A$ pointwise fixes $X$ setwise.
\end{proposition}

\begin{definition}
    An $\mathcal{L}_0$-structure $\mathcal{N}$ is \emph{definably interpreted} in an $\mathcal{L}$-structure $\mathcal{M}$ 
    if and only if one can find a definable $X \subseteq M^n$ for some $n$ and can interpret the symbols of $\mathcal{L}_0$
    as definable subsets and functions on $X$ (definable using $\mathcal{L}$)
    so that the resulting $\mathcal{L}_0$-structure is isomorphic to $\mathcal{N}$.

    An $\mathcal{L}_0$-structure $\mathcal{N}$ is \emph{interpretable} in an $\mathcal{L}$-structure $\mathcal{M}$
    if there is a definable $X \subseteq M^n$, a definable equivalence relation $E$ on $X$,
    and for each symbol of $\mathcal{L}_0$, one can find definable $E$-invariant sets on $X$
    such that $X / E$ with the induced structure is isomorphic to $\mathcal{N}$.
\end{definition}

Let $S$ be a set.
The universe of a \emph{many-sorted structure} $\mathcal{N}$ with sorts $S$ is 
a set $N$ that is partitioned into disjoint sets $\{N_i : i \in S\}$.
For each $n$-ary relation symbol $R$,
there are $s_1,\dots,s_n \in S$ such that $R^{\mathcal{N}} \subset N^{s_1} \times \dots \times N^{s_n}$.
For each $n$-ary function symbol $f$,
there are $s_0,\dots,s_n \in S$ such that $f^{\mathcal{N}} : N^{s_1} \times \dots \times N^{s_n} \to N^{s_0}$.

Let $\mathcal{M}$ be an $\mathcal{L}$-structure.
The set of sorts $S = \{S_E : E$ an $\emptyset$-definable equivalence relation on $M^n$ for some $n$ $\}$.
In the many-sorted structure $\mathcal{M}^{\mathrm{eq}}$,
one interpret the sort $S_E$ as $M^n / E$ for $E$ an $\emptyset$-definable equivalence relation on $M^n$.
Because $=$ is a definable equivalence relation on $M$,
$M$ can be identified with the sort $S_{=}$.
All relations and functions of $\mathcal{L}$ are relations and functions on $M^k$.
For each $\emptyset$-definable equivalence relation $E$ on $M^n$,
we have in $\mathcal{M}^{\mathrm{eq}}$ an $n$-ary function $f_E: M^n \to S_E$ given by $f_E(\overline{x}) = \overline{x} / E$.

\begin{lemma}
    \begin{enumerate}[label = {\roman*)}]
        \item If $X \subseteq M^n$ is definable in $\mathcal{M}^{\mathrm{eq}}$, then $X$ is definable in $\mathcal{M}$.
        \item If $\mathcal{M} \equiv \mathcal{N}$, then $\mathcal{M}^{\mathrm{eq}} \equiv \mathcal{N}^{\mathrm{eq}}$.
        \item If $\sigma$ is an automorphism of $\mathcal{M}^{\mathrm{eq}}$, then $\sigma|M$ is an automorphism of $\mathcal{M}$.
        \item If $\sigma$ is an automorphism of $\mathcal{M}$,
          there is $\widehat{\sigma}$ an automorphism of $\mathcal{M}^{\mathrm{eq}}$ such that $\sigma = \widehat{\sigma}|M$.
    \end{enumerate}
\end{lemma}

\subsection{Basic Techniques and Important Results}
\subsubsection{Completeness and Compactness}
Let $T$ be an $\mathcal{L}$-theory and $\phi$ an $\mathcal{L}$-sentence.
A proof of $\phi$ from $T$ is a finite sequence of $\mathcal{L}$-formulas
$\psi_1,\dots,\psi_m$ such that $\psi_m = \phi$ and $\psi_i \in T$ or 
$\psi_i$ follows from $\psi_1,\dots,\psi_{i-1}$ by a simple logical rule for each $i$.

Denote $T \vdash \phi$ if there is a proof of $\phi$ from $T$.

Some properties of the proof system:
\begin{enumerate}[label = {$\bullet$}]
    \item Proofs are finite.
    \item (Soundness) If $T \vdash \phi$, then $T \models \phi$.
    \item If $T$ is a finite set of sentences, then there is an algorithm that,
      when given a sequence of $\mathcal{L}$-formulas $\sigma$ and an $\mathcal{L}$-sentence $\phi$,
      will decide whether $\sigma$ is a proof of $\phi$ from $T$.
\end{enumerate}

A language $\mathcal{L}$ is \emph{recursive} if there is an algorithm that
decides whether a sequence of symbols is an $\mathcal{L}$-formula.
An $\mathcal{L}$-theory $T$ is recursive if there is a algorithm that,
when given an $\mathcal{L}$-sentence $\phi$ as input, decides whether $\phi \in T$.

\begin{theorem}[G\"odel's Completeness Theorem]
    Let $T$ be an $\mathcal{L}$-theory and $\phi$ an $\mathcal{L}$-sentence, then $T \models \phi$ if and only if $T\vdash \phi$.
\end{theorem}

\begin{definition}
    An $\mathcal{L}$-theory $T$ is \emph{inconsistent} 
    if $T \vdash (\phi \land\neg\phi)$ for some sentence $\phi$;
    otherwise $T$ is \emph{consistent}.
\end{definition}

\begin{theorem}[Compactness Theorem]
    $T$ is satisfiable if and only if every finite subset of $T$ is satisfiable.
\end{theorem}

\begin{definition}
    A theory $T$ is \emph{finitely satisfiable} if every finite subset of $T$ is satisfiable.
\end{definition}

% \begin{definition}
%     An $\mathcal{L}$-theory $T$ has the \emph{witness property} if 
%     whenever $\phi(v)$ is an $\mathcal{L}$-formula with one free variable $v$,
%     then there is a constant symbol $c \in \mathcal{L}$ such that 
%     $T \models (\exists v\, \phi(v) \to \phi(c))$.
% \end{definition}

\begin{definition}
    An $\mathcal{L}$-theory $T$ is \emph{maximal} if for all $\phi$ either $\phi \in T$ or $\neg \phi \in T$. 
\end{definition}

% \begin{lemma}
%     Suppose that $T$ is a maximal and finitely satisfiable $\mathcal{L}$-theory with witness property.
%     Then, $T$ has a model.
%     In fact, if $\kappa$ is a cardinal and $\mathcal{L}$ has at most $\kappa$ constant symbols, then there exists $\mathcal{M} \models T$ with $|\mathcal{M}| \le \kappa$.
% \end{lemma}

% \begin{lemma}
%     Let $T$ be a finitely satisfiable $\mathcal{L}$-theory.
%     There is a language $\mathcal{L}^* \supseteq \mathcal{L}$ and $T^* \supseteq T$ a finitely satisfiable $\mathcal{L}^*$-theory such that any $\mathcal{L}^*$-theory extending $T^*$ has the witness property.
%     $\mathcal{L}^*$ can be chosen such that $|\mathcal{L}^*| = |\mathcal{L}| + \aleph_0$.
% \end{lemma}

\begin{lemma}
    Suppose that $T$ is a finitely satisfiable $\mathcal{L}$-theory and $\phi$ is an $\mathcal{L}$-sentence, then,
    either $T \cup \{\phi\}$ or $T \cup \{\neg\phi\}$ is satisfiable.
\end{lemma}

\begin{corollary}
    If $T$ is a finitely satisfiable $\mathcal{L}$-theory,
    then there is a maximal finitely satisfiable $\mathcal{L}$-theory $T' \supseteq T$.
\end{corollary}

\begin{theorem}
    If $T$ is a finitely satisfiable $\mathcal{L}$-theory and $\kappa$ is an infinite cardinal with $\kappa \ge |\mathcal{L}|$,
    then there is a model of $T$ of cardinality at most $\kappa$. 
\end{theorem}

\begin{lemma}
    If $T \models \phi$, then $\Delta \models \phi$ for some finite $\Delta \subseteq T$.
\end{lemma}

\begin{definition}
    An $\mathcal{L}$-theory $T$ is called \emph{complete} if for any $\mathcal{L}$-sentence $\phi$,
    either $T \models \phi$ or $T \models \neg\phi$.
\end{definition}

\begin{proposition}
    Let $T$ be an $\mathcal{L}$-theory with infinite models.
    If $\kappa$ is an infinite cardinal and $\kappa \ge |\mathcal{L}|$,
    then there is a model of $T$ of cardinality $\kappa$.
\end{proposition}

\begin{definition}
    Let $\kappa$ be an infinite cardinal and let $T$ be a theory with models of size $\kappa$.
    $T$ is \emph{$\kappa$-categorical} if any two models of $T$ of cardinal $\kappa$ are isomorphic.
\end{definition}

\begin{tcolorbox}
Some abbreviation for commonly used theories:
\begin{enumerate}[label = {}]
    \item \textbf{DLO}: dense linear order;
    \item \textbf{ACF}: algebraic closed field;
    \item \textbf{RCF}: real closed field;
    \item \textbf{DCF}: differential closed field.
\end{enumerate}
\end{tcolorbox}

Let $\mathbf{ACF_p}$ be the theory of algebraically closed fields of characteristic $p$,
where $p$ is either $0$ or a prime number.

\begin{proposition} 
    $\mathbf{ACF}_p$ is $\kappa$-categorical for all uncountable cardinals $\kappa$.
\end{proposition}

\begin{theorem}[Vaught's Test]
    Let $T$ be a satisfiable theory with no finite models that is 
    $\kappa$-categorical for some infinite cardinal $\kappa \ge |\mathcal{L}|$.
    Then $T$ is complete.
\end{theorem}

\begin{definition}
    An $\mathcal{L}$-theory $T$ is \emph{decidable} if there is an algorithm that
    when given an $\mathcal{L}$-sentence $\phi$ as input decides whether $T \models \phi$.
\end{definition}

\begin{lemma}
    Let $T$ be a recursive complete satisfiable theory in a recursive language $\mathcal{L}$.
    Then $T$ is decidable.
\end{lemma}

\begin{corollary}
    For $p = 0$ or $p$ prime, $\mathbf{ACF}_p$ is decidable.
\end{corollary}

\begin{corollary}
    Let $\phi$ be a sentence in the language of rings.
    The following are equivalent.
    \begin{enumerate}[label = {\roman*)}]
        \item $\phi$ is true in the complex numbers.
        \item $\phi$ is true in every algebraically closed field of characteristic zero.
        \item $\phi$ is true in some algebraically closed field of characteristic zero.
        \item There are arbitrarily large primes $p$ such that $\phi$ is true in some algebraically closed field of characteristic $p$.
        \item There is an $m$ such that for all $p > m$, $\phi$ is true in all algebraically closed fields of characteristic $p$.
    \end{enumerate}
\end{corollary}

\subsubsection{Between structures}
\begin{definition}
    If $\mathcal{M}$ and $\mathcal{N}$ are $\mathcal{L}$-structures,
    then an $\mathcal{L}$-embedding $j: \mathcal{M} \to \mathcal{N}$ is called an \emph{elementary embedding} if
    \[
    \mathcal{M} \models \phi(a_1,\dots,a_n) \iff \mathcal{N} \models \phi(j(a_1),\dots,j(a_n))
    \] 
    for all $\mathcal{L}$-formulas $\phi(v_1,\dots,v_n)$ and all $a_1,\dots,a_n \in M$.

    If $\mathcal{M}$ is a substructure of $\mathcal{N}$, 
    it is called an \emph{elementary substructure} and 
    denote $\mathcal{M} \prec \mathcal{N}$ if the inclusion map is elementary,
    equivalently, $\mathcal{N}$ is an \emph{elementary extension} of $\mathcal{M}$.
\end{definition}

\begin{theorem}[Upward L\"owenheim--Skolem Theorem]
    Let $\mathcal{M}$ be an infinite $\mathcal{L}$-structure and 
    $\kappa$ be an infinite cardinal $\kappa \ge |\mathcal{M}| + |\mathcal{L}|$.
    Then, there is $\mathcal{N}$ an $\mathcal{L}$-structure of cardinality $\kappa$ and $j; \mathcal{M} \to \mathcal{N}$ is elementary.
\end{theorem}

\begin{proposition}[Tarski--Vaught Test]
    Suppose that $\mathcal{M}$ is a substructure of $\mathcal{N}$.
    Then, $\mathcal{M}$ is an elementary substructure if and only if,
    for any formula $\phi(v,\overline{w})$ and $\overline{a} \in M$,
    if there is $b \in N$ such that $\mathcal{N} \models \phi(b,\overline{a})$,
    there is $c \in M$ such that $\mathcal{N} \models \phi(c,\overline{a})$.
\end{proposition}

An $\mathcal{L}$-theory $T$ has \emph{built-in Skolem functions}
if for all $\mathcal{L}$-formulas $\phi(v,w_1,\dots,w_n)$ there is a function symbol $f$
such that $T \models \forall \overline{w} ((\exists v \phi(v,\overline{w})) \to \phi(f(\overline{w}),\overline{w}))$.
In other words, there are enough function symbols in the language to witness all existential statements.

\begin{lemma}
    Let $T$ be an $\mathcal{L}$-theory.
    There are $\mathcal{L}^* \supseteq \mathcal{L}$and $T^* \supseteq T$ an $\mathcal{L}^*$-theory
    such that $T^*$ has built-in Skolem functions,
    and if $\mathcal{M} \models T$,
    then $\mathcal{M}$ can be expanded to $\mathcal{M}^* \models T^*$.
    $\mathcal{L}^*$ can be chosen such that $|\mathcal{L}^*|= |\mathcal{L}|+\aleph_0$.

    We call $T^*$ a skolemization of $T$.
\end{lemma}

\begin{theorem}[L\"owenheim--Skolem Theorem]
    Suppose $\mathcal{M}$ is an $\mathcal{L}$-structure and
    $X \subseteq M$, there is an elementary submodel $\mathcal{N}$ of $\mathcal{M}$
    such that $X \subseteq N$ and $|\mathcal{N}| \le |X| + |\mathcal{L}| + \aleph_0$.
\end{theorem}

\begin{definition}
    A \emph{universal sentence} is one of the form
    $\forall \overline{v} \phi(\overline{v})$,
    where $\phi$ is quantifier-free.
    An $\mathcal{L}$-theory $T$ has a \emph{universal axiomatization} if there is a set of universal $\mathcal{L}$-sentences $\Gamma$
    such that $\mathcal{M} \models \Gamma$ if and only if $\mathcal{M} \models T$ for all $\mathcal{L}$-structures $\mathcal{M}$.
\end{definition}

\begin{theorem}
    An $\mathcal{L}$-theory $T$ has a universal axiomatization if and only if
    whenever $\mathcal{M} \models T$ and $\mathcal{N}$ is a substructure of $\mathcal{M}$,
    then $\mathcal{N} \models T$.
\end{theorem}

\begin{definition}
    Suppose that $(I,<)$ is a linear order.
    Suppose that $\mathcal{M}_i$ is an $\mathcal{L}$-structure for $i \in I$.
    $(\mathcal{M}_i: i \in I)$ is a \emph{chain} of $\mathcal{L}$-structures if $\mathcal{M}_i \subseteq \mathcal{M}_j$ for $i < j$.
    If $\mathcal{M}_i \prec \mathcal{M}_j$ for $i < j$,
    $(\mathcal{M}_i: i \in I)$ is an \emph{elementary chain}.
\end{definition}

\begin{proposition}
    Suppose that $(I,<)$ is a linear order and
    $(\mathcal{M}_i: i \in I)$ is an elementary chain.
    Then, $\mathcal{M} = \bigcup_{i\in I} M_i$ is an elementary extension of each $\mathcal{M}_i$.
\end{proposition}

\begin{definition}
    A theory $T$ has a $\forall\exists$-axiomatization if it can be axiomatized by sentences of
    the form $\forall v_1\dots\forall v_n \exists w_1\dots \exists w_m \phi(\overline{v},\overline{w})$
    where $\phi$ is a quantifier-free formula.
\end{definition}

\begin{definition}
    $\mathcal{M} \models T$is existentially closed
    if whenever $\mathcal{N} \models T$, $\mathcal{N} \supseteq \mathcal{M}$,
    and $\mathcal{N} \models \exists \overline{v}\, \phi(\overline{v},\overline{a})$,
    where $\overline{a} \in M$ and $\phi$ is quantifier-free,
    then $\mathcal{M} \models \exists \overline{v} \, \phi(\overline{v},\overline{a})$.
\end{definition}

\begin{definition}[Ultrafilters]
    Let $I$ be a set and $\mathcal{P}(I) = \{ X : X \subset I\}$
    denote the power set of $I$.
    A \emph{filter} on $I$ is a collection $D \subset \mathcal{P}(I)$ such that:
    \begin{enumerate}[label = {\roman*)}]
        \item $I \in D$, $\emptyset \not\in D$;
        \item if $A$, $B \in D$, then $A \cap B \in D$;
        \item if $A\in D$ and $A \subseteq B \subseteq I$, then $B \in D$.
    \end{enumerate}
    Examples:
    \begin{enumerate}
        \item Let $\kappa$ be an infinite cardinal with $\kappa \le |I|$. 
        $D = \{ X \subseteq I : |I \backslash X| < \kappa\}$ is a filter.
        If $\kappa = \aleph_0$, $D$ is called the \emph{Fr\'echet filter}.
        \item For $x \in I$, $D = \{X\subseteq I: x\in X\}$ is a filter on $I$, called the \emph{principal filter}.
    \end{enumerate}

    A filter $D$ on $I$ is an \emph{ultrafilter} if $X \in D$ or $I\backslash X \in D$ for all $X \subseteq I$.
\end{definition}

\begin{definition}[Ultraproducts]
    Let $\mathcal{L}$ be a language and suppose that $I$ is an infinite set.
    Suppose that $\mathcal{M}_i$ is an $\mathcal{L}$-structure for each $i \in I$.
    Let $D$ be an ultrafilter on $I$.
    Define a new structure $\mathcal{M} = \prod \mathcal{M}_i / D$,
    which is called the \emph{ultraproduct} of the $\mathcal{M}_i$ using $D$.
    Define a relation $\sim$ on 
    \[
    X = \prod_{i\in I}M_i = \Bigl\{ f: I \to \bigcup_{i\in I}M_i : f(i)\in M_i \text{ for all $i$} \Bigr\}
    \]
    by $f\sim g$ if and only if $\{i \in I:f(i) = g(i)\} \in D$.

    The universe of $\mathcal{M}$ will be $M = X / \sim$.
    
    If $c$ is a constant symbol of $\mathcal{L}$,
    let $c^{\mathcal{M}}$ be the $\sim$ class of $f_c \in X$ where $f_c(i) = c^{\mathcal{M}_i}$ for all $i\in I$. 

    If $f$ is an $n$-ary function symbol of $\mathcal{L}$.
    Suppose that $g_1,\dots,g_n,h_1,\dots,h_n\in X$,
    and $g_i \sim h_i$ for $i = 1, \dots,n$.
    Define $g_{n+1}(i) = f^{\mathcal{M}_i}(g_1(i),\dots,g_n(i))$ and 
    $h_{n+1}(i) = f^{\mathcal{M}_i}(h_1(i),\dots,h_n(i))$ for $i \in I$.
    Then one can show $g_{n+1} \sim h_{n+1}$.
    Thus $f^{\mathcal{M}}(g_1/\sim,\dots,g_n/\sim) = g_{n+1}/\sim$ determines a well-defined function on $\mathcal{M}$.

    Suppose that $R$ is a relation symbol of $\mathcal{L}$ and
    $g_1,\dots,g_n,h_1,\dots,h_n\in X$ as above.
    Then $\{i\in I: (g_1(i),\dots,g_n(i)) \in R^{\mathcal{M}_i}\} \in D$ if and only if
    $\{i \in I: (h_1(i),\dots,h_n(i)) \in R^{\mathcal{M}_i}\} \in D$.
    One can interpret
    \[
    R^{\mathcal{M}} = \{(g_1/\sim,\dots,g_m/\sim): \{ i\in I: (g_1(i),\dots,g_m(i)) \in R^{\mathcal{M}_i}\} \in D\}.
    \]
\end{definition}

\begin{definition}
    Let $\mathcal{M}$ be a fixed $\mathcal{L}$-structure,
    and $\mathcal{M}_i = \mathcal{M}$ for every $i \in \omega$.
    Let $D$ be a non-principal ultrafilter on $\omega$.
    Let $\mathcal{M}^* = \prod \mathcal{M}_i / D$,
    which is called an \emph{ultrapower} of $\mathcal{M}$.
\end{definition}

\begin{theorem}[Keisler--Shelah Theorem]
    Two $\mathcal{L}$-structures $\mathcal{M}$ and $\mathcal{N}$ are elementarily equivalent
    if and only if there is an index set $I$ and an ultrafilter $D$ on $I$ such that $\prod \mathcal{M}/D \cong \prod \mathcal{N}/D$.
\end{theorem}

\subsection{Algebraic Examples and Quantifiers}
The study of definable sets is often made quite complicated by quantifiers.

\begin{definition}
    A theory $T$ has \emph{quantifier elimination} if for every formula $\phi$ there is a quantifier-free formula $\psi$ such that
    \[
    T \models \phi \leftrightarrow \psi.
    \]
\end{definition}

\begin{definition}
    An $\mathcal{L}$-theory $T$ is \emph{strongly minimal} if for any $\mathcal{M} \models T$
    every definable subset of $M$ is either finite or cofinite.
\end{definition}

\begin{definition}
    If $T$ is a theory then $T_{\forall}$ is the set of all universal consequences of $T$.
    $\mathcal{A} \models T_{\forall}$ if and only if there is $\mathcal{M} \models T$ with $\mathcal{A} \subseteq \mathcal{M}$.

    A theory $T$ has \emph{algebraically prime models}
    if for any $\mathcal{A} \models T_{\forall}$ there is 
    $\mathcal{M} \models T$ and an embedding $i:\mathcal{A} \to \mathcal{M}$
    such that for all $\mathcal{N} \models T$ and embedding $j: \mathcal{A} \to \mathcal{N}$
    there is $h: \mathcal{M} \to \mathcal{N}$ such that $j = h \circ i$.

    If $\mathcal{M}$, $\mathcal{N} \models T$ and $\mathcal{M} \subseteq \mathcal{N}$,
    $\mathcal{M}$ is \emph{simply closed} in $\mathcal{N}$ and denote
    $\mathcal{M} \prec_s \mathcal{N}$ if for any quantifier free formula
    $\phi(\overline{v},w)$ and any $\overline{a} \in M$,
    if $\mathcal{N} \models \exists w\, \phi(\overline{a},w)$,
    then so does $\mathcal{M}$.
\end{definition}

\begin{definition}
    An $\mathcal{L}$-theory $T$ is \emph{model-complete} if,
    $\mathcal{M} \prec \mathcal{N}$ whenever
    $\mathcal{M} \subseteq \mathcal{N}$  and $\mathcal{M}$, $\mathcal{N} \models T$.
\end{definition}

\begin{proposition}
    If $T$ has quantifier eliminations,
    then $T$ is model-complete.
\end{proposition}

\begin{proposition}
    Let $T$ be a model-complete theory. 
    Suppose that there is $\mathcal{M}_0 \models T$ such that $\mathcal{M}_0$ embeds into every model of $T$. 
    Then, $T$ is complete.
\end{proposition}

\begin{definition}
    An ordered structure $(M,<,\dots)$ is \emph{o-minimal} if
    for any definable $X \subset M$,
    there are finitely many intervals $I_1,\dots,I_m$ with endpoints in $M \cup \{\pm\infty\}$
    and a finite set $X_0$ such that $X = X_0 \cup I_1 \cup \dots \cup I_m$.
\end{definition}

\subsubsection{Algebraic Closed Fields}

% \begin{lemma}
%     $\mathbf{ACF}_{\forall}$ is the theory of integral domains.
% \end{lemma}

\begin{theorem}
    $\mathbf{ACF}$ has quantifier elimination.
\end{theorem}

\begin{corollary}
    $\mathbf{ACF}$ is model-complete and $\mathbf{ACF}_p$ is complete where $p = 0$ or $p$ is prime.
\end{corollary}

Let $K$ be a field.
$X \subseteq K^n$ is \emph{Zariski closed} if $X = V(S)$ for some $S \subseteq K[X_1,\dots,X_n]$.

\begin{lemma}
    Let $K$ be a field.
    The subsets of $K^n$ defined by atomic formulas are exactly those of the form $V(p)$ for some $p \in K[\overline{X}]$.
    A subset of $K^n$ is quantifier-free definable if and only if it is a Boolean combination of Zariski closed sets.
\end{lemma}

\begin{corollary}
    Let $K$ be an algebraically closed field.
    \begin{enumerate}[label = {\roman*)}]
        \item $X \subset K^n$ is constructable if and only if it is definable.
        \item (Chevalley's Theorem) The image of a constructable set under a polynomial map is constructable.
    \end{enumerate}
\end{corollary}

\begin{corollary}
    $\mathbf{ACF}$ is a strongly minimal theory.
\end{corollary}

\subsubsection{Real Closed Fields}

\begin{definition}
    A field $F$ is \emph{orderable} if there is a linear order $<$ of $F$ making $(F,<)$ an ordered field.

    F is \emph{formally real} if $-1$ is not a sum of squares.
\end{definition}

\begin{theorem}
    If $F$ is a formally real field, then $F$ is orderable.
\end{theorem}

\begin{definition}
    A field $F$ is \emph{real closed} if it is formally real with no proper formally real algebraic extensions.
\end{definition}

\begin{theorem}
    Let $F$ be a formally real field. The following are equivalent.
    \begin{enumerate}[label = {\roman*)}]
        \item $F$ is real closed.
        \item $F(i)$ is algebraically closed.
        \item For any $a\in F$, either $a$ or $-a$ is a square and every polynomial of odd degree has a root.
    \end{enumerate}
\end{theorem}

\begin{corollary}
    The class of real closed field is an elementary class of 
    $\mathcal{L}_r$-structures.
\end{corollary}

\begin{definition}
    Let $\mathbf{RCF}$ be the $\mathcal{L}_{or}$-theory axiomatized by the axioms for real closed fields
    and the axioms for ordered fields.

    If $F$ is a formally real field, a \emph{real closure} of $F$ is a real closed algebraic extension of $F$.
\end{definition}

\begin{theorem}
    If $(F,<)$ is an ordered field,
    and $R_1$ and $R_2$ are real closures of $F$ where the canonical ordering extends the ordering of $F$,
    then there is a unique field isomorphism $\phi: R_1 \to R_2$ that is identity on $F$.
\end{theorem}

\begin{corollary}
    $\mathbf{RCF}$ has algebraically prime models.
\end{corollary}

\begin{theorem}
    The theory $\mathbf{RCF}$ admits elimination of quantifiers in $\mathcal{L}_{or}$.
\end{theorem}

\begin{corollary}
    $\mathbf{RCF}$ is complete, model complete, and decidable.
\end{corollary}

\subsection{Semialgebraic Sets and O-minimality}
\begin{definition}
    Let $F$ be an ordered field.
    $X \subset F^n$ is \emph{semialgebraic} if it is a Boolean combination of sets of the form $\{\overline{x}:p(\overline{x})>0\}$,
    where $p(\overline{X}) \in F[X_1,\dots,X_n]$.
\end{definition}

\begin{corollary}[Tarski--Seidenberg Theorem]
    The semialgebraic sets are closed under projection.
\end{corollary}

\begin{corollary}
    If $F \models \mathbf{RCF}$ and $A \subseteq F^n$ is semialgebraic,
    then the closure (topological) of $F$ is semialgebraic.
\end{corollary}

A function is semialgebraic if its graph is semialgebraic.

\begin{definition}
    Let $F$ be a real closed field and
    $f(\overline{X}) \in F(X_1,\dots,X_n)$ be a rational function.
    $f$ is \emph{positive semi-definite} if $f(\overline{a}) \ge 0$ for all $\overline{a} \in F^n$.
\end{definition}

\begin{theorem}[Hilbert's 17th Problem]
    If $f$ is a positive semi-definite rational function over a real closed field $F$,
    then $f$ is a sum of squares of rational functions.
\end{theorem}

\begin{corollary}
    $\mathbf{RCF}$ is an o-minimal theory.
\end{corollary}

\begin{lemma}
    If $f:\mathbb{R} \to \mathbb{R}$ is semialgebraic,
    then for any open interval $U \subseteq \mathbb{R}$,
    there is a point $x\in U$ such that $f$ is continuous at $x$.
\end{lemma}

\begin{corollary}
    Let $F$ be a real closed field and $f: F \to F$ is a semialgebraic function.
    Then, one can partition $F$ into $I_1\cup \dots \cup I_m\cup X$,
    where $X$ is finite and the $I_j$ are pairwise disjoint open intervals with endpoints in $F\cup \{\pm\infty\}$
    such that $f$ is continuous on each $I_j$.
\end{corollary}

\begin{corollary}[Curve Selection]
    Let $F$ be a real closed field.
    Let $X \subseteq F^n$ be semialgebraic and $\overline{a}$
    be a point in the closure of $X$.
    There is a continuous semialgebraic function $f:(0,r) \to F^n$
    such that for all $\epsilon\in(0,r)$, $f(x) \in X$ and $\lim_{\epsilon\to 0} f(\epsilon) = \overline{a}$.
\end{corollary}

\begin{corollary}
    Let $F$ be a real closed field.
    Let $E \subseteq F^n \times F^n$ be a definable equivalence relation.
    There is a definable $X \subset F^n$ such that for all 
    $a \in F^n$ there is a unique $b\in X$ such that $aEb$.
    Such $X$ is called a definable \emph{transversal} of $E$.
\end{corollary}

\begin{definition}[Cell]
    Inductive definition of the collection of \emph{cells}.
    \begin{enumerate}[label = {$\bullet$}]
        \item $X \subseteq F^n$ is a $0$-cell if it is a single point.
        \item $X \subset F$ is a $1$-cell if it is an interval $(a,b)$,
          where $a \in F \cup \{-\infty\}$, $b\in F\cup \{+\infty\}$, and $a < b$.
        \item If $X \subset F^n$ is an $n$-cell and $f: X\to F$ is a continuous definable function,
          then $Y = \{(\overline{x},f(\overline{x})): \overline{x} \in X\}$ is an $n$-cell.
        \item Let $X \subseteq F^n$ be an $n$-cell. Suppose that $f$ is
          either a continuous definable function from $X$ to $F$ or identically $-\infty$ and
          $g$ is either a continuous definable function from $X$ to $F$
          such that $f(\overline{x}) < g(\overline{x})$ for all $\overline{x} \in X$ or identically $+infty$, then 
        \[
        Y = \{(\overline{x},y): \overline{x} \in X \land f(\overline{x}) < y < g(\overline{x})\}
        \]
        is an $n+1$-cell.
    \end{enumerate}
\end{definition}

\begin{lemma}
    Let $X \subseteq F^{n+1}$ be semialgebraic.
    There is a natural number $N$ such that if $\overline{a}\in F^n$
    and $X_{\overline{a}} = \{y: (\overline{a},y)\in X\}$ is finite,
    then $|X_{\overline{a}}| < N$.
\end{lemma}

\begin{theorem}[Cell Decomposition]
    Let $X \subseteq F^m$ be semialgebraic.
    There are finitely many pairwise disjoint cells $C_1,\dots,C_n$ such that $X = C_1 \cup \dots \cup C_n$.
\end{theorem}

\begin{theorem}[Real Nullstellensatz]
    Let $F$ be a real closed field,
    and let $I$ be an ideal in $F[\overline{X}]$.
    Then, $V_F(I)$ is nonempty if and only if whenever $p_1,\dots,p_m\in F[\overline{X}]$ and $\sum p_i^2 \in I$,
    then all the $p_i \in I$. 
\end{theorem}

Let $\mathbb{R}_{\exp} = (\mathbb{R},+,\cdot,\exp,<,0,1)$.

\begin{theorem}
    The theory of $\mathbb{R}_{\exp}$ is model-complete and o-minimal.
\end{theorem}

% \subsubsection{van den Dries's original statements}

% We work with a fixed but arbitrary o-minimal structure $(R,<,\mathcal{S})$.

% \begin{theorem}[Monotonicity Theorem]
%     Let $f: (a,b)\to R$ be a definable function on the interval $(a,b)$.
%     Then there are points $a_1 < \cdot < a_k$ in $(a,b)$ such that on each subinterval $(a_j,a_{j+1})$,
%     with $a_0 = a$, $a_{k+1} = b$,
%     the function is either constant, or strictly monotone and continuous.
% \end{theorem}

% \begin{corollary}
%     Let $f: (a,b) \to R$ be definable.
%     Then for each $c\in (a,b)$ the limit $\lim_{x\uparrow c}f(x)$
%     and $\lim_{x\downarrow c}f(x)$ exist in $R_{\infty}$.
%     Also the limits $\lim_{x\uparrow b}f(x)$ and $\lim_{x\downarrow a}f(x)$ exist in $R_{\infty}$.
% \end{corollary}

% \begin{corollary}
%     Let $f: [a,b]\to R$ be continuous and definable.
%     Then $f$ takes a maximum and a minimum value on $[a,b]$.
% \end{corollary}

% \begin{lemma}[Finiteness Lemma]
%     Let $A \subset R^2$ be definable and suppose that for each $x\in R$
%     the fiber $A_x \coloneq \{y\in R\colon (x,y)\in A\}$ is finite.
%     Then there is $N \in \mathbb{N}$ such that $A_x \le N$ for all $x\in R$.
% \end{lemma}

% A decomposition $\mathcal{D}$ of $R^m$ is said to \textbf{partition}
% a set $S \subseteq R^m$ if each cell in $\mathcal{D}$ is
% either part of $S$ or disjoint from $S$,
% in other words, if $S$ is a union of cells in $\mathcal{D}$.
% \begin{theorem}[Cell Decomposition]
%     \hfill
%     \begin{enumerate}[label = {$\mathrm{(\Roman*_{m})}$}]
%         \item Given any definable sets $A_1,dots,A_k \subset R^m$ there is a decomposition of $R^m$ partitioning each of $A_1,\dots,A_k$.
%         \item For each definable function $f: A \to R$, $A \subseteq R^m$,
%         there is a decomposition $\mathcal{D}$ of $R^m$ partitioning $A$ such that 
%         the restriction $f|B: B \to R$ to each cell $B \in \mathcal{D}$ with $B \subseteq A$ is continuous.
%     \end{enumerate}
% \end{theorem}

% \begin{theorem}
%     The theory $\mathbb{R}_{\exp}$ is model-complete and o-minimal.
% \end{theorem}

% \subsubsection{}
% \begin{theorem}
%   Let $(R,<,\mathcal{S})$ be an o-minimal structure and 
%   $S \subseteq R^{p+q}$ a definable set.
%   Then there is a positive integer $d = d(S)$ such that
%   for all sufficiently large $n\in\mathbb{N}$
%   each $n$-element set $F \subseteq R^q$ has at most $n^d$ subsets of the form $S_x \cap F$ with $x \in R^p$.
% \end{theorem}

% Let $\mathcal{C}$ be a collection of subsets of an infinite set $X$.
% Given $F \subseteq X$,
%   \[
%     \mathcal{C}\cap F := \{\mathcal{C}\cap F:C \in \mathcal{C}\},
%   \]
% the set of intersections of sets in $\mathcal{C}$ with $F$.
%   If $A \subseteq F$ is of the form $A = C\cap F$ for some $C\in \mathcal{C}$,
%   $A$ is \emph{cut out from $F$ by a set in $\mathcal{C}$}.
%   \[
%     f_{\mathcal{C}}(n):=\max\{|\mathcal{C}\cap F|:F\text{ is an $n$-element subset of $X$}\}.
%   \]

% \begin{theorem}
%   Either $f_{\mathcal{C}}(n) = 2^n$ for all $n$,
%   or else there is $d\in \mathbb{N}$ such that 
%   $f_{\mathcal{C}}(n) = n^d$ for all sufficiently large $n$.
% \end{theorem}

% \begin{theorem}
%   Let $\mathcal{R} = (R,\dots)$ be an infinite model-theoretic structure and suppose 
%   all definable relations $\Phi \subseteq R^p \times R$ for all $p > 0$, and dependent.
%   Then all definable relations $\Phi \subseteq R^p \times R^q$,
%   for all $p$, $q > 0$, are dependent.
% \end{theorem}

% \begin{theorem}[Ramsey's Theorem]
%   Given positive integers $M$, $r$ and $k$,
%   there is a positive integer $N = N(M,r,k)$ so large that 
%   if $X$ is a set with $|X| \ge N$ and
%   $X^{(r)} = P_1 \cup P_2 \cup \dots \cup P_k$,
%   then there is a set $Y \subseteq X$ with $|Y|=M$ such that 
%   $Y^{(r)} \subseteq P_j$ for some $j\in \{1,\dots,k\}$.
% \end{theorem}

% \begin{definition}
%   Let $X$ be an infinite set.
%   Given a relation $A \subseteq X^r$,
%   a finite sequence $x_1,\dots,x_M$ in $X$ is \emph{$A$-indiscernible} if 
%   \[
%     (x_{i(1)},\dots,x_{i(r)})\in A \iff (x_{j(1)},\dots,x_{j(r)}) \in A.
%   \]
%   whenever $1\le i(1) < \cdots < i(r) \le M$ and
%   $1\le j(1) < \cdots < j(r) \le M$.
%   Let $\mathcal{A}$ be a collection of relations on $X$,
%   that is, each element of $\mathcal{A}$ is a set $A \subseteq A^r$,
%   with $r \in\mathbb{N}$ depending on $A$.
%   Then a sequence $x_1,\dots,x_M$ in $X$ is called \emph{$\mathcal{A}$-indiscernible} if it is $A$-indiscernible for each $A \in \mathcal{A}$.
%   A \emph{subsequence} of a sequence $x_1,\dots,x_N$ is by definition a sequence $x_{i(1)},\dots,x_{i(M)}$ with 
%   $1\le i(1)< \dots < i(M) \le N$.
% \end{definition}

% \begin{corollary}
%     Let $X$ be an infinite set and $\mathcal{A}$ a finite collection of relation on $X$.
%     Then, given any positive integer $M$ there is a positive integer $N$ so large that each sequence in $X$ of length $N$ contains an $\mathcal{A}$-indiscernible subsequence of length $M$.
% \end{corollary}

% TODO

% \subsubsection{}
% Fix an o-minimal expansion $(R,<,\mathcal{S})$ of an ordered abelian group $(R,<,0,-,+)$.

% Definable curves play a role similar to that of sequences in $\mathbb{R}$, better properties.

% \begin{corollary}[Curve Selection]
%     If $a\in\mathrm{cl}(X)-X$,
%     where $X$ is definable,
%     then there is a definable continuous injective map
%     $\gamma: (0,\epsilon) \to X$, for some $\epsilon > 0$,
%     such that $\lim_{t\to 0}\gamma(t) = a$.
% \end{corollary}

% \begin{lemma}
%     Let $f: X \to R^n$ be a definable continuous map on a closed bounded set $X \subseteq R^m$.
%     Then $f(X)$ is bounded in $R^n$.
% \end{lemma}

% \begin{proposition}
%     If $f: X \to R^n$ is a continuous definable map on a closed bounded set $X \subseteq R^m$,
%     then $f(X)$ is closed and bounded in $R^m$.
% \end{proposition}

% TODO

% \subsubsection{}
% \begin{theorem}[Triangulation Theorem]
%     Let $S \subset R^m$ be a definable set,
%     with definable subsets $S_1,\dots,S_k$.
%     Then $S$ has a triangulation in $R^m$ that is compatible with these subsets.
% \end{theorem}

% TODO

\subsection{About Types}
Suppose that $\mathcal{M}$ is an $\mathcal{L}$-structure and $A \subseteq M$.
Let $\mathcal{L}_A$ be the language obtained by adding to $\mathcal{L}$ constant symbols for each $a\in A$.
We can naturally view $\mathcal{M}$ as an $\mathcal{L}_A$-structure by interpreting the new symbols in the obvious way.
Let $\mathrm{Th}_A(\mathcal{M})$ be the set of all $\mathcal{L}_A$-sentences true in $\mathcal{M}$.
% Note that $\mathrm{Th}_A(\mathcal{M}) \subseteq \mathrm{Diag_el}(\mathcal{M})$.

\begin{definition}
    Let $p$ be the set of $\mathcal{L}_A$-formulas in free variables $v_1,\dots,v_n$.
    $p$ is an \emph{$n$-type} if $p\cup \mathrm{Th}_A(\mathcal{M})$ is satisfiable.
    $p$ is a \emph{complete $n$-type} if $\phi\in p$ or $\neg\phi\in p$ for all $\mathcal{L}_A$-formulas $\phi$ with free variables from $v_1,\dots,v_n$.
    Let $S^{\mathcal{M}}_n(A)$ be the set of all complete $n$-types. 

    If $p$ is an $n$-type over $A$,
    $\overline{a}\in M^n$ \emph{realizes $p$} if $\mathcal{M} \models \phi(\overline{a})$ for all $\phi\in p$.
    If $p$ is not realized in $\mathcal{M}$,
    then $\mathcal{M}$ \emph{omits $p$}.
\end{definition}

TODO

\begin{definition}
    Let $\kappa$ be an infinite cardinal.
    $\mathcal{M} \models T$ is $\kappa$-saturated if,
    for all $A \subseteq M$, if $|A| < \kappa$ and $p \in S^{\mathcal{M}}_n(A)$,
    then $p$ is realized in $\mathcal{M}$.

    $\mathcal{M}$ is \emph{saturated} if it is $|M|$-saturated.
\end{definition}

TODO

\begin{proposition}
    Let $\mathcal{M}$ be saturated.
    Let $A \subset M$ with $|A| < |M|$.
    Let $X \subset M^n$ be definable with parameters from $M$.
    Then, $X$ is $A$-definable if and only if every automorphism of $\mathcal{M}$ that fixes $A$ pointwise fixes the $X$ setwise.
\end{proposition}

\subsection{Definable Spaces and Quotients}
\subsubsection{Definable spaces}
Let a covering $S = \bigcup_i U_i$ of a set $S$ by subsets $U_i$ ($i\in I$) be given,
and for each index $i \in I$ a set-theoretic bijection $g_i: U_i \to U'_i$ a topological space,
such that the transition maps are continuous.

Suppose that the index set $I$ is finite,
that each $U'_i \subseteq R^{m(i)}$ is a definable set,
and for each pair $i$, $j$ the set $g_i(U_i\cap U_j) \subseteq U'_i$
is definable and $g_{ij}: g_i(U_i\cap U_j) \to g_j(U_j\cap U_i)$ is definable.
(Such a family is called a \emph{definable atlas of $S$}.)

Let $X \subseteq S$ and $f : X \to R$;
call $X$ \emph{definable} if $g_i(X\cup U_i) \subseteq I'_i$ is definable for each $i$,
and call $f$ \emph{definable} if $X$ is definable and 
$f_i : g_i(X \cap U_i) \to R$ given by $f_i(x) = f(g_i^{-1}(x))$ is definable for each $i$.
If $X = X_1\cup \cdots \cup X_n$,
where all $X_k$ are definable,
then a function $f: X \to R$ is definable if and only if each restriction $f|X_k: X_k \to R$ is definable.

Note that $U_i$ is definable,
and for $g_i = (g_{i1},\dots,g_{im(i)}) : U_i \to R^{m(i)}$,
each $g_{ik}$ is definable.
Let $\mathbf{DO}(S)$ be the collection of definable open subsets of $S$,
and for each $U \in \mathbf{DO}(S)$,
let $\mathbf{DC}(U)$ be the $R$-algebra of definable continuous functions $f: U \to R$.
Then we call the set $S$ equipped with the sheaf $(\mathbf{DC}(U))_{U \in \mathbf{DO}(S)}$ a \emph{definable space}.

\begin{definition}
    Let $S$ and $T$ be definable spaces.
    Then a map $F: S \to T$ is said to be \emph{definable} if its graph $\Gamma(F)$ is a definable subset of
    the product space $S \times T$.
\end{definition}

\begin{definition}
    Recall that a topological space $S$ is said to be regular if for each $a \in  S$ and
    open $U \subseteq S$ with $a \in U$ there is an open $V \subseteq S$ with $a \in V$ and $\mathrm{cl}(V) \subseteq U$.
    A definable space $S$ is regular if and only if for each $a \in S$ and definable open $U \subseteq S$ with $a \in U$
    there is a definable open $V \subseteq S$ with $a \in V$ and $\mathrm{cl}(V) \subseteq U$,
    and also if and only if for each $a \in S$ and definable closed $X \subseteq S$ with $a \notin X$
    there are disjoint definable open neighborhoods of $a$ and $X$ in $S$.
\end{definition}

\subsubsection{Definable quotient spaces}

Let $E \subseteq X\times X$ be a definable equivalence relation on a definable set $X \subseteq R^m$.

\begin{definition}
    A \emph{definable quotient of $X$ be $E$} is a pair $(p,Y)$ consisting of a definable set $Y \subseteq R^n$ and definable continuous surjective map $p: X\to Y$ such that 
    \begin{enumerate}[label = {(\roman*)}]
        \item $E = E_p$, i.e., $(x_1,x_2) \in E \iff p(x_1) = p(x_2)$ for all $x_1$, $x_2 \in X$;
        \item $p$ is definably identifying, i.e., for all definable $K \subseteq Y$, if $p^{-1}(K)$ is closed in $X$, then $K$ is closed in $Y$.
    \end{enumerate}
\end{definition}

Let $E$ be a definable equivalence relation on a definable set $X$.
Let $pr_1 : E \to X$ and $pr_2 : E \to X$ be the restrictions of the two projection maps $X\times X \to X$.

$E$ is called \emph{definably proper over $X$} if $pr_1$ is a definably proper map.

\begin{theorem}
    Suppose the definable equivalence relation $E$ on the definable set $X$ is definably proper over $X$.
    Then $X/E$ exists as a definably proper quotient of $X$.
\end{theorem}

\section{Definable complex geometry}
Materials follow from \cite{MR2441378}.

Let $\mathbf{R}$ be a real closed field and $\mathbf{K}$ its algebraic closure,
identified with $\mathbf{R}^2$ (after fixing a square-root of $-1$).

\begin{definition}
    A \emph{definable $C^0$ $\mathbf{R}$-manifold of dimension $n$, with respect to $\mathbf{R}$} is a set $X$,
    covered by finitely many nonempty sets $U_1, \dots, U_k$,
    and for each $i = 1, \dots, k$,
    there is a set-theoretic bijection $\phi_i:U_i \to V_i$,
    where $V_i$ is definable and open in $\mathbf{R}^n$ and such that each $\phi_i(U_j\cap U_i)$ is definable an open
    and the transition maps are definable and continuous.
    Moreover, the topology induced on $X$ by this covering is Hausdorff.

    A manifold is a \emph{$C^p$ $\mathbf{R}$-manifold} if in addition
    the transition maps are $C^p$ with respect to the field $\mathbf{R}$.
\end{definition}

\begin{definition}
    Let $X$ be a definable subset of $\mathbf{R}^n$.
    $X$ is called \emph{definably compact} if for every definable continuous $\gamma:(0,1) \to X$ the limit of $\gamma(t)$,
    as $t$ tends to $0$ in $\mathbf{R}$, exists in $X$.
    This is equivalent to $X$ being closed and bounded in $\mathbf{R}^n$.

    A definable set $X \subseteq \mathbf{R}^n$ is \emph{locally definably compact}
    if every $x \in $ has a definable neighborhood $V\subseteq X$
    (i.e., $V$ contains an $X$-open set around $x$)
    which is definably compact.

    $X \subseteq \mathbf{R}^n$ is \emph{locally closed} if there is a (definable) open set $U \subseteq \mathbf{R}^n$ containing $X$
    such that $X$ is relatively closed in $U$.

    Let $U$ be an open subset of $\mathbf{R}^n$.
    For $X \subseteq U$,
    the \emph{frontier of $X$ in $U$} is defined as $Fr_U(X) = Cl_U(X)\backslash X$,
    where $Cl_U(X)$ is the closure of $X$ in $U$.
    If $U = \mathbf{R}^n$ then we write $Fr(X)$ instead of $Fr_{\mathbf{R}^n}$.  
\end{definition}

\begin{lemma}
    Let $X$ be a definable subset of $\mathbf{R}^n$.
    Then the following are equivalent:
    \begin{enumerate}[label = {(\roman*)}]
        \item $X$ is locally definably compact.
        \item $X$ is locally closed in $\mathbf{R}^n$.
        \item $Fr(X)$ is closed subset of $\mathbf{R}^n$.
    \end{enumerate}
\end{lemma}

\begin{definition}
  Let $f$ be a definable continuous map from a definable $X \subseteq \mathbf{R}^n$ into $Y \subseteq \mathbf{R}^k$.

  For $b \in Y$, $f$ is \emph{definably proper over $b$} if for every definable curve $\gamma:(0,1) \to X$ such that $\lim_{t\to 0} f(\gamma(t)) = b$,
  $\gamma(t)$ tends to some limit in $X$ as $t$ tends to $0$.
  If $f: X \to Y$ is definably proper over every $b\in Y$ then
  we say that $f$ is \emph{definably proper over $Y$} or just 
  $f$ is \emph{definably proper}.

  For $A \subseteq X$, we say that $f|A$ is \emph{definably proper over its image} if $f|A : A \to f(A)$ is definably proper over $f(A)$.

  We say that $f$ is \emph{bounded over $b\in Y$} if there is a neighborhood $W \subseteq Y$ of $b$ such that $f^{-1}(W)$ is bounded subset of $\mathbf{R}^n$. 
\end{definition}

\begin{lemma}
  For $f: X \to Y$ a definable continuous map,
  $X \subseteq \mathbf{R}^n$,
  and $y\in Y$, the following are equivalent:
  \begin{enumerate}[label = {(\roman*)}]
    \item $f$ is definably proper over $y$.
    \item $f$ is bounded over $y$ and the intersection of the closure of the graph of $f$ 
      in $\mathbf{R}^n\times Y$ with $Fr(X)\times \{y\}$ is empty.
  \end{enumerate}
\end{lemma}

\begin{lemma}
  Let $X \subseteq \mathbf{R}^n$ be a definable,
  locally closed set, $f: X \to \mathbf{R}^k$ a definable
  continuous map. Then,
  \begin{enumerate}[label = {(\roman*)}]
    \item The set of all $y \in \mathbf{R}^k$ such that $f$ is definably proper over $y$ is open in $\mathbf{R}^k$.
    \item If $f$ is definably proper over $f(X)$, then $f(X)$ is locally closed set.
  \end{enumerate}
\end{lemma}

TODO

\begin{theorem}[Definable Chow's Theorem]
  Let $A$ be a definable $\mathbf{K}$-algebraic subset of $\mathbf{K}^n$.
  Then $A$ is an algebraic set over $\mathbf{K}$.
\end{theorem}

\begin{theorem}[The dimension Theorem]
  Let $M$ be an $n$-dimensional $\mathbf{K}$-manifold,
  $X$, $Y$ definable, irreducible $\mathbf{K}$-analytic subsets of $M$.
  Then every irreducible component of $X \cap Y$ has $\mathbf{K}$-dimension not less than $\dim_{\mathbf{K}}X + \dim_{\mathbf{K}}Y - n$.
\end{theorem}

\begin{theorem}[Remmert'd proper mapping theorem]
  Let $f: M \to N$ be a $\mathbf{K}$-holomorphic map between $\mathbf{K}$-manifolds.
  Let $A \subseteq M$ be a $\mathbf{K}$-analytic subset.
  Assume that $f(A)$ is a closed subset of $N$.
  Then $f(A)$ is a $\mathbf{K}$-analytic subset of $N$.
\end{theorem}

\section{O-minimal GAGA Results}
O-minimal GAGA theorems are from \cite{zbMATH07662555},
which are used in the proof of the boundedness in \cite{arXiv:2507.00973,arXiv:2508.19215}
to adapt to the non-algebraic period domains from moduli spaces of abelian varieties.

\begin{tcolorbox}[title = {\bfseries\Large Main results}]
\begin{theorem}[{Definable GAGA, \cite[Theorem 2.1]{arXiv:2508.19215},\cite[Theorem 1.4]{zbMATH07662555}}]
    Le $X$ be an algebraic space and $X^{\definable}$ the associated definable analytic space.
    The definabilization functor $\mathrm{Coh}(X) \to \mathrm{Coh}(X^{\definable})$ is fully faithful, exact and 
    its essential image if closed under subobjects and quotients.
\end{theorem}

\begin{theorem}[{Definable images, \cite[Theorem 2.2]{arXiv:2508.19215},\cite[Theorem 1.3]{zbMATH07662555}}]
    \label{def image}
    Let $X$ be an algebraic space and 
    $\phi: X^{\definable} \to \mathcal{Z}$ a proper morphism of definable analytic spaces.
    Then there is a factorization
    \[\begin{tikzcd}
    X^{\definable}\ar[rr,"\phi"]\ar[dr,"f^{\definable}"'] & & \mathcal{Z} \\
    & Y^{\definable}\ar[ur,"\iota"'] &
    \end{tikzcd}\]
    where $f: X\to Y$ is a proper dominant morphism of algebraic spaces and
    $\iota: Y^{\definable} \to \mathcal{Z}$ is a closed embedding of definable analytic spaces.
    Moreover, $f$ is uniquely determined as a morphism with fixed source.
\end{theorem}
\end{tcolorbox}

The following materials follow from \cite{zbMATH07662555}.

\subsection{Definable Analytic Spaces}
\subsubsection{Local case}
Identify $\mathbb{C} = \mathbb{R}^2$ using the real and imaginary parts,
and give $\mathbb{C}$ the definable structure coming from the identification $\mathbb{C}^n = \mathbb{R}^{2n}$.
For a definable open $U \subset \mathbb{C}^n$,
let $\mathcal{O}_{\mathbb{C}^n}(U)$ be the definable holomorphic functions on $U$,
that is, the maps $U \to \mathbb{C}$ that are both definable and holomorphic.

\begin{definition}[Definition 2.19]
    Given an open definable subset $U \subset \mathbb{C}^n$ and 
    a finitely generated ideal $I$ of $\mathcal{O}_{\mathbb{C}^n}(U)$,
    the vanishing locus $X = |V(I)|$ is naturally a definable topological space.
    We call the data of $U \subset \mathbb{C}^n$ and $I$ a \emph{basic definable complex analytic space}.
    We often refer to the basic definable complex analytic space via $X \subset U \subset \mathbb{C}^n$,
    and denote by $I_X := I\mathcal{O}_U$.
\end{definition}

\begin{theorem}[Definable Oka coherence, Theorem 2.21]
    The definable structure sheaf $\mathcal{O}_{\mathbb{C}^n}$ of $\mathbb{C}^n$ is a coherent $\mathcal{O}_{\mathbb{C}^n}$-module.
\end{theorem}

Given a basic definable complex analytic space $X \subset U \subset \mathbb{C}^n$,
we may naturally consider $X$ as an analytic space,
which is denoted $X^{\analytic}$.
For simplicity, denote $\mathbf{Coh}(X) := \mathbf{Coh}(\mathcal{O}_X)$ and
$\mathbf{Coh}(X^{\analytic}) := \mathbf{Coh}(\mathcal{O}_{X^{\analytic}})$.
There is a natural morphism $g: (X^{\analytic},\mathcal{O}_{X^{\analytic}}) \to (\underline{X},\mathcal{O}_X)$
of locally $\mathbb{C}$-ringed sites,
and a resulting analytification functor $(-)^{\analytic}: \mathbf{Coh}(X) \to \mathbf{Coh}(X^{\analytic})$
given by $F^{\analytic} := \mathcal{O}_{X^{\analytic}} \otimes_{g^{-1}\mathcal{O}_X} g^{-1}F$ together with
a natural identification $\mathcal{O}_X^{\analytic} \cong \mathcal{O}_{X^{\analytic}}$.

\begin{theorem}[Theorem 2.27]
    Let $X$ be a basic definable complex analytic space and
    $(-)^{\analytic} : \mathbf{Coh}(X) \to \mathbf{Coh}(X^{\analytic})$ the analytification functor.
    Then
    \begin{enumerate}[label = {(\arabic*)}]
        \item $(-)^{\analytic}$ is exact;
        \item $(-)^{\analytic}$ is faithful.
    \end{enumerate}
\end{theorem}

\subsubsection{Global case}
\begin{definition}[Definition 2.35]
    A locally $\mathbb{C}$-ringed definable space $(X,\mathcal{O}_X)$ is \emph{locally a basic definable complex analytic space}
    if on a definable cover it is isomorphic to the locally $\mathbb{C}$ definable space
    associated to a basic definable complex analytic space.
    We define the category of \emph{definable complex analytic spaces} (DefAnSp/$\mathbb{C}$) to be
    the full subcategory of the category of locally $\mathbb{C}$-ringed definable spaces consisting of $(X,\mathcal{O}_X)$
    which are locally a basic definable complex analytic space.
\end{definition}

\begin{theorem}[Theorem 2.38]
    Let $X$ be a definable complex analytic space.
    Then $\mathcal{O}_X$ is a coherent $\mathcal{O}_X$-module.
\end{theorem}

\begin{theorem}[Theorem 2.39]
    Let $X$ be a definable complex analytic space.
    Then the analytification functor 
    $(-)^{\analytic}: \mathbf{Coh}(X) \to \mathbf{Coh}(X^{\analytic})$ is exact and faithful.
\end{theorem}

\begin{corollary}[Corollary 2.40]
    For $X$ a definable complex analytic space,
    a sequence $M' \to M \to M''$ of coherent $\mathcal{O}_X$-modules
    is exact if and only if it is exact on stalks (or even analytic stalks).
\end{corollary}

\subsubsection{Reduced case}
\begin{proposition}[Proposition 2.45]
    Let $X$ be a definable complex analytic space and 
    $\mathcal{Y} \subset X$ a closed definable complex analytic subset.
    Then $\mathcal{Y}$ canonically has the structure of a reduced closed definable complex analytic subspace $Y \subset X$.
\end{proposition}

\subsubsection{Indcution method}
\begin{proposition}[Definable Noetherian induction, Proposition 2.46]
    Let $X$ be a definable complex analytic space and $F$ a coherent sheaf on $X$. 
    Any increasing chain of coherent subsheaves of $F$ must stabilize.
\end{proposition}

\subsubsection{Analytic factorization}
\begin{proposition}[Proposition 2.55]
    Let $X$, $Y$, $Z$ be definable complex analytic spaces and
    suppose we have (solid) diagrams
    \[
    \begin{tikzcd}
        X \ar[r,"h"] \ar[d,"g"'] & Y\ar[dl,"f",dotted] \\ Z 
    \end{tikzcd}
    \hspace*{30pt}
    \begin{tikzcd}
        X^{\analytic} \ar[r,"h^{\analytic}"] \ar[d,"g^{\analytic}"'] & Y^{\analytic}\ar[dl,"\varphi"] \\ Z^{\analytic}
    \end{tikzcd}
    \]
    such that $h$ is proper, surjective on points,
    and $\mathcal{O}_Y \to h_*\mathcal{O}_X$ is injective.
    Then a unique $f$ exists such that $f^{\analytic} = \varphi$.
\end{proposition}

\subsection{Definable GAGA Theorem}

\begin{theorem}[Theorem 3.1]
    \label{def GAGA}
    Let $X$ be an algebraic space and
    \[(-)^{\definable} : \mathbf{Coh}(X) \to \mathbf{Coh}(X^{\definable})\]
    the definabilization functor.
    Then 
    \begin{enumerate}[label = {(\arabic*)}]
        \item $(-)^{\definable}$ is fully faithful and exact.
        \item The essential image of $(-)^{\definable}$ is closed under taking subobjects and quotients.
    \end{enumerate}
\end{theorem}

\begin{theorem}[{Definable Chow theorem, Theorem 3.3, Peterzil--Strarchenko \cite[Corollary 4.5]{zbMATH05364146}}]
    Let $Y$ be a reduced algebraic space and 
    $\mathcal{X} \subset Y^{\definable}$ a closed definable complex analytic subset.
    Then $\mathcal{X}$ is algebraic.
\end{theorem}

\begin{lemma}[Lemma 3.4]
    $(-)^{\definable}$ is faithful and exact.
\end{lemma}

\begin{lemma}[Lemma 3.5]
    \label{Lemma 3.5}
    Let $X$ be an algebraic space.
    Then Definable GAGA Theorem \ref{def GAGA} holds for $X$ if and only if 
    for every algebraic coherent sheaf $F$ on $X$,
    any definable coherent subsheaf $\mathcal{E} \subset F^{\definable}$ is the definabilization of
    an algebraic coherent subsheaf $E \subset F$.
\end{lemma}

The two observations imply that Theorem \ref{def GAGA} holds for $X$
if and only if it holds on the reduction $X^{\mathrm{red}}$:
\begin{lemma}[Lemma 3.6]
    \label{Lemma 3.6}
    Let $X$ be an algebraic space with a nilpotent sheaf of ideals $I$ cutting out a subspace $X_0$.
    Then Theorem \ref{def GAGA} holds for $X_0$ if and only if it holds for $X$.
\end{lemma}

\begin{proof}[Proof of Theorem \ref{def GAGA}]
    Proceed by Noetherian induction on $X$,
    assuming that Theorem \ref{def GAGA} holds for every proper subspace of $X$.
    By \ref{Lemma 3.6}, $X$ can be assumed reduced.
    Let $F$ be an algebraic coherent sheaf on $X$ and $\mathcal{E} \subset F^{\definable}$ a definable coherent subsheaf.
    By Lemma \ref{Lemma 3.5} we must show that $\mathcal{E}$
    is the definabilization of an algebraic coherent subsheaf $E \subset F$.

    \begin{enumerate}[label = {Step \arabic*.}]
        \item
        \begin{lemma}[Lemma 3.7]
            Any exact sequence 
            \[
            0 \to \mathcal{E} \to F^{\definable} \to \mathcal{G} \to 0
            \]
            in $\mathbf{Coh}(X^{\definable})$ for which $\mathcal{E}$ and $\mathcal{G}$ are locally free
            in the definabilization of an exact sequence
            \[
            0 \to E \to F \to G \to 0
            \]
            in $\mathbf{Coh}(X)$ where $E$ and $G$ are locally free.
        \end{lemma}

        \item
        \begin{lemma}[Lemma 3.8]
            For some dense open $U \subset X$, $\mathcal{E}_{|U}$ is algebraic.
        \end{lemma}

        \item
        With the notation above, 
        let $E_U$ be the algebraic sheaf on $U$ for which $(E_U)^{\definable} \cong \mathcal{E}_{|U}$.
        Let $\widetilde{E}$ be the "closure" of $E_U$ in $F$, i.e.,
        the pullback 
        \[\begin{tikzcd}
            F \ar[r] & j_*j^*F \\
            \widetilde{E} \ar[u]\ar[r] & j_*E_U \ar[u]
        \end{tikzcd}\]
        where $j: U \hookrightarrow X$ denotes the inclusion.
        The sheaf $\widetilde{E}$ is evidently quasi-coherent and so it is coherent since it is a subsheaf of $F$.
        Thus, $\widetilde{E}^{\definable}$ and $\mathcal{E}$ are both definable coherent subsheaves of $F^{\definable}$,
        and therefore so is their intersection $\mathcal{G}$.

        Let $I_Z$ be the ideal sheaf of $Z = X \backslash U$ with the reduced algebraic space structure,
        and $\mathcal{I} = I_Z^{\definable}$.
        \begin{lemma}[Lemma 3.9]
            Suppose we have definable coherent sheaves $\mathcal{G} \subset \mathcal{G}'$
            for which $\mathcal{G}_{|U} = \mathcal{G}'_{|U}$.
            Then for some positive integer $n$, $\mathcal{I}^n\mathcal{G}' \subset \mathcal{G}$.
        \end{lemma}
        Applying the lemma to $\mathcal{G} \subset \widetilde{E}^{\definable}$,
        we have $(I^n_Z\widetilde{E})^{\definable} \subset \mathcal{E}$ for some positive integer $n$.
        The quotient $\mathcal{E}'$ is then a subsheaf of $(F')^{\definable}$,
        where $F' = F/I^n_Z\widetilde{E}$ is supported on a subspace whose reduction is $Z$.
        By the induction hypothesis, $\mathcal{E}'$ is algebraic,
        and $\mathcal{E}$ is the preimage in $F$,
        hence algebraic, so the proof is complete.
    \end{enumerate}
\end{proof}

\begin{corollary}[Corollary 3.10]
    Let $Y$ be an algebraic space and $\mathcal{X} \subset Y^{\definable}$ a closed definable complex analytic subspace.
    Then $\mathcal{X}$ is (uniquely) the definabilization of an algebraic subspace.
\end{corollary}

\begin{corollary}[Corollary 3.11]
    Let $X$, $Y$ be algebraic spaces.
    Then any morphism $X^{\definable} \to Y^{\definable}$ of definable complex analytic space is (uniquely)
    the definabilization of an algebraic morphism.
\end{corollary}

\subsection{Definable Images}

\begin{definition}
    A morphism $f:X \to Y$ of algebraic spaces is dominant if $\mathcal{O}_Y \to f_*\mathcal{O}_X$ is injective.
\end{definition}

Goal:
\begin{theorem}
    Let $X$ be an algebraic space,
    $\mathcal{S}$ a definable complex analytic space,
    and $\varphi: X^{\definable} \to \mathcal{S}$ a proper definable complex analytic morphism.
    Then there exists a (unique) factorization
    \[\begin{tikzcd}
        X^{\definable}\ar[dr,"f^{\definable}"']\ar[rr,"\varphi"] & & \mathcal{S} \\
        & Y^{\definable} \ar[ur,"\iota",hook] & 
    \end{tikzcd}\]
    where $f: X \to Y$ is dominant algebraic and $\iota$ is a definable closed immersion.
    Moreover, $\iota^{\analytic}(Y^{\analytic})$ coincides with the image $\varphi^{\analytic}(X^{\analytic})$.
\end{theorem}

The proof involves two main steps:
the case that $X$ is reduced and the reduction to this case.

\begin{proposition}[Proposition 4.5]
    Let $f: W \to Z$ be a proper dominant morphism of algebraic spaces.
    Suppose we have an algebraic square-zero thickening
    $W \to W'$, a definable closed immersion $Z^{\definable} \to \mathcal{Z}'$,
    and a morphism $\varphi': W^{\definable}\to \mathcal{Z}'$ which fits into a commutative diagram
    \[\begin{tikzcd}
        W^{\definable} \ar[r]\ar[d,"f^{\definable}"'] & W^{'\definable}\ar[d,"\varphi'"] \\
        Z^{\definable} \ar[r] & \mathcal{Z}'
    \end{tikzcd}\]
    Then the following are uniquely defined:
    an algebraic square-zero thickening
    $Z \to Z''$,
    a definable closed immersion $Z^{''\definable} \to \mathcal{Z}'$,
    and a (proper) dominant morphism $f': W' \to Z''$ of algebraic spaces,
    such that we have commutative diagrams
    \[
    \begin{tikzcd}
        W\ar[r]\ar[d,"f"'] & W'\ar[d,"f'"] \\
        Z \ar[r] & Z''
    \end{tikzcd}
    \hspace{40pt}
    \begin{tikzcd}
        W^{'\definable}\ar[dd,"f^{'\definable}"']\ar[rd,"\varphi'"] & \\
        & \mathcal{Z}' \\
        Z^{''\definable}\ar[ur] &
    \end{tikzcd}
    \]
\end{proposition}

\begin{proposition}[Proposition 4.6]
    Let $X$ be a smooth algebraic space,
    $\mathcal{U}$ a smooth definable complex analytic space,
    and $\varphi: X^{\definable} \to \mathcal{U}$ a smooth proper definable analytic morphism.
    Then $\varphi: X^{\definable} \to \mathcal{U}$ is the definabilization of an algebraic morphism $f: X \to U$ and
    the associated morphism $U \to \mathrm{Hilb}(X)$ is a closed embedding.
\end{proposition}

\begin{lemma}[Lemma 4.11]
    Suppose $Z$ is an algebraic space,
    and $J$ is a coherent sheaf on $Z$.
    Let $R$ be a sheaf of rings on the {\'e}tale site $Z^{\text{\' et}}$ of $Z$ such that 
    \[
    0 \to J \to R \to \mathcal{O}_Z \to 0
    \]
    is a first order thickening.
    Then $(Z^{\text{\'et}},R)$ is an algebraic space.
\end{lemma}

\emergencystretch = 1em
\printbibliography

\end{document}
